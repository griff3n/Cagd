%Qt Neuland
%Qt Designer nicht gut in VS eingebunden
%Debugging in MS VS
%Performance
%Geeignete Architektur

Bei der Arbeit am Projekt sind wir nur auf wenige grö\ss{}ere Probleme gesto\ss{}en. 
Die meisten Probleme, die wir hatten, hatten mit der Fehlerfindung in Visual Studio und C++ zu tun. 
An vielen Stellen war es nicht offensichtlich, an welcher Stelle nun ein Fehler auftrat und warum. 

Das Problem wurde insbesondere dadurch verstärkt, dass wir keinerlei Vorkenntnis mit QT hatten, welches wir für die UI nutzen wollten. 
Dies sollte uns einiges an Arbeit im Bereich der Oberflächengestaltung ersparen.
Im Endeffekt haben wir vermutlich das gleiche an Einarbeitungszeit aufwenden müssen. 
Des Weiteren hatten wir einige Probleme mit dem QT Designer Tool, welches schnelles und einfaches erstellen von UIs ermöglichen soll. 
Insbesondere dessen Visual Studio Einbindung funktioniert nicht immer, wie wir es erwartet hätten. 

Ein weiteres Problem war das Finden einer geeigneten und erweiterbaren Architektur für das Projekt. 
Insbesondere die Erweiterbarkeit war zu Anfang schwierig, da noch nicht alle Aufgaben und nötigen Funktionen bekannt waren.
Dementsprechend durchlief die Architektur einige Iterationen.

Das letzte gro\ss{}e Problem war die Performance. 
Mit wachsendem Funktionsumfang wurde das Projekt leider deutlich langsamer. 
Bei ``grö\ss{}eren`` Meshes -viele Vertices und Edges- dauerten dann Catmull-Clark Iterationen eine Minute oder länger, bei einem Model mit 50.000 Vertices und 300.000 Halfedges dauert selbst das Laden des Objektes etwa 30 Sekunden.
