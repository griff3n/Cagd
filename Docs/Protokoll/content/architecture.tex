\subsection{Vorgaben}
Es galt eine Architektur aufzubauen, die auf einer HalfEdge-Datenstruktur basiert. Das hei\ss{}t insbesondere:
\begin{itemize}
\item Jeder Punkt kennt genau \textit{eine} seiner ausgehenden und \textit{keine} seiner eingehenden Kanten.
\item Jede Kante kennt nur ihren Ausgangspunkt, ihre Gegenkante und die nachfolgende Kante.
\item Jede Fläche kennt genau \textit{eine} ihrer umschlie\ss{}enden Kanten. 
\item Bei einem zusammenhängenden Mesh ist jeder Punkt von jedem anderen über die Kanten erreichbar.
\item Gehört eine Fläche nicht zum Mesh wird diese als ``Hole`` bezeichnet.
\end{itemize}
In den folgenden Abschnitten wird die erstellte Architektur dargestellt und kurz erklärt.

\subsection{Architektur}
Die Architektur des Projektes teilt sich im wesentlichen in ein Model- und ein View-Paket. 
Während das Model eine etwas erweiterte HalfEdge-Struktur bereitstellt, ist das View-Paket im wesentlichen für die Anzeige zuständig. 

Im folgenden soll zuerst das Model-Paket und anschlie\ss{}end das View-Paket vorgestellt werden.

\subsubsection{Model}
\begin{figure}[htbp]
\includegraphics[angle=90,scale=0.5]{content/pictures/architekturModel.png}
\caption{Model-Paket der Architektur}
\label{fig:model}
\end{figure}

Wie bereits erwähnt stellt das Model in Abb.~\ref{fig:model} unter anderem die HalfEdge-Struktur bereit. 
Hierzu dienen die Strukturen graphicVertex, halfEdge und graphicFace, welche alle von der abstrakten GraphicObject-Klasse erben. 
Dies erleichtert später die Verknüpfung der verschiedenen Level of Detail (LOD) nach Ausführungen von Unterteilungsalgorithmen. 
Zur einfacheren Datenhaltung werden die genannten Strukturen als HalfEdgeMesh-Klasse gespeichert, die alle Punkte, Kanten und Flächen eines Objektes enthält. 
Des Weiteren kennt das Mesh sein LOD. 
Die genannten Objekte werden über die ObjectLoader-Klasse geladen.\\

Die ObjectMemory-Klasse ist zurzeit noch ungenutzt. 
Mit Verweis auf Kapitel~\ref{chap:Ausblick} sollte an dieser Stelle z.B. eine Undo-Redo-Funktion realisiert werden. 
Dies wurde aufgrund von Zeitmangel jedoch nicht umgesetzt.

\subsubsection{View}
\begin{figure}[htbp]
\includegraphics[angle=90,scale=0.6]{content/pictures/architekturView.png}
\caption{View-Paket der Architektur}
\label{fig:view}
\end{figure}

Um den Aufwand für UI-Programmierung zu reduzieren, haben wir uns des QT-Frameworks bedient. 
Dem entsprechend besteht die View zum grö\ss{}ten Teil aus externen QT-Abhängigkeiten. 
Das Hauptfenster der Anwendung wird in QTCagd realisiert, welche eine mit dem QT-Designer erstellte QTCagd.ui einbindet und anzeigt (siehe Abb.~\ref{fig:view}).
Des Weiteren werden dort entsprechende Signale für UI-Eingaben und Auswahlen gesendet.

Das OpenGLWidget hingegen ist ein Teil des Hauptfensters und kümmert sich um sämtliche OpenGL-spezifischen Anzeigen und Operationen.
Dies umfasst z.B. das Anzeigen von Vertices, Edges und Faces. 

Das Skin-Paket umfasst Objekte, die zum Anzeigen der internen GraficObjects genutzt werden. 
So existiert z.B ein Sphere-Model, das genutzt wird, um dem Nutzer die Auswahl von Vertices zu erleichtern. 
Der ContextSensitiveEditView ist das Menü auf der rechten Seite in Abb.~\ref{fig:ui}. Dieses Menü ändert sich basierend auf der momentanen Auswahl.

\begin{figure}[htbp]
\centering
\includegraphics[scale=0.7]{content/pictures/cadgWindow.PNG}
\caption{UI der Anwendung}
\label{fig:ui}
\end{figure}