Die Anwendung könnte auf mehrere Arten erweitert werden.
Zum einen wäre es möglich weitere Unterteilungsverfahren zu implementieren, wie z.B. Loop-Subdivide oder Doo-Sabin.
Andererseits wäre es natürlich auch möglich bestehende Funktionen auszubauen.
So könnte auch eine einfache Rotation des Meshes statt der Kamera stattfinden.
Des Weiteren wäre die bereits genannte Undo-Redo-Funktion eine sinnvolle Erweiterung für die Anwendung.
Au\ss{}erdem könnte man noch eine einzige Fläche zur weiteren Verarbeitung (verschieben, löschen, scharf setzen,...) unterteilen lassen, statt das ganze Mesh zu unterteilen.
Eine abschlie\ss{}ende Erweiterungsmöglichkeit wäre eine Extrude-Funktion, die es dem Nutzer erlaubt eine Fläche aus einer existierenden zu extrahieren, welche das Mesh fortsetzt.

Im Bereich der Optimierung liegt ganz klar die Performance im Fokus. 
Diese könnte vermutlich mit Techniken wie asynchroner Verarbeitung und Multithreading deutlich verbessert werden.
